\documentclass[11pt]{scrreprt}

\usepackage{cmap}

\usepackage[utf8]{inputenc}
\usepackage[T1]{fontenc}
\usepackage[english, ngerman]{babel}

\usepackage{microtype}
\usepackage{csquotes}
\usepackage{enumitem}
\usepackage{amsmath}
\usepackage{amsthm}
\usepackage{amsfonts}
\usepackage{mathtools}
\usepackage{xpatch}
\makeatletter
\xpatchcmd{\proof}{\@addpunct{.}}{\@addpunct{:}}{}{}
\makeatother

\setlength{\parindent}{0cm}
\newcounter{thm}
\numberwithin{thm}{section}

\newtheorem{definition}[thm]{Definition}
\newtheorem{satz}[thm]{Satz}
\newtheorem{korollar}[thm]{Korollar}
\newtheorem{uebung}[thm]{Übung}
\newtheorem{beispiel}[thm]{Beispiel}
\newtheorem{lemma}[thm]{Lemma}
\newtheorem*{bemerkung*}{Bemerkung}
\newtheorem*{definition*}{Definition}
\begin{document}

\chapter*{I. Unrestringierte Probleme}
\setcounter{chapter}{1}
\setcounter{section}{1}

\section{Lösbarkeit}

\setcounter{thm}{2}

\begin{definition}[Lösbarkeit] ~\\
	Das Minimierungsproblem $P$ heißt \textbf{lösbar}, falls ein $\overline{x} \in M$ existiert mit
	$$ \inf_{x \in M} f(x) = f(\overline{x}) $$
\end{definition}

\setcounter{thm}{4}

\begin{satz}
	Das Minimierungsproblem $P$ ist genau dann lösbar, wenn es einen globalen Minimalpunkt besitzt.
\end{satz}

\begin{bemerkung*}
	Es können drei Fälle der Unlösbarkeit auftreten:

	\begin{itemize}
		\item $\inf_{x \in M} f(x) = + \infty$
		\item $\inf_{x \in M} f(x) = - \infty$
		\item Ein endliches Infimum wird nicht angenommen.
	\end{itemize}	
\end{bemerkung*}


\begin{satz}[Satz von Weierstraß] ~\\
	Die Menge $M \subseteq \mathbb{R}^n$ sei nichtleer und kompakt, und die Funktion $f \colon M \rightarrow \mathbb{R}$ sei stetig. Dann besitzt $f$ auf $M$ (mindestens) einen globalen Minimalpunkt und einen globalen Maximalpunkt.
\end{satz}

\setcounter{thm}{7}

\begin{definition}[Untere Niveaumenge]
	Für $X \subseteq \mathbb{R}^n, f \colon X \rightarrow \mathbb{R}$ und $\alpha \in \mathbb{R}$ heißt
	$$ \operatorname{lev}_{\leq}^{\alpha}(f, X) = \big\{ x \in X ~|~ f(x) \leq \alpha \big\} $$
	\textbf{untere Niveaumenge von $f$ auf $X$ zum Niveau $\alpha$}. Im Fall $X = \mathbb{R}^n$ schreiben wir auch kurz
	$$ f_{\leq}^{\alpha} \coloneqq \operatorname{lev}_{\leq}^{\alpha}(f, \mathbb{R}^n) = \big\{ x \in \mathbb{R}^n ~|~ f(x) \leq \alpha \big\} $$
\end{definition}

\setcounter{thm}{9}

\begin{uebung}
	Für eine abgeschlossene Menge $X \subseteq \mathbb{R}^n$ sei die Funktion $f \colon X \rightarrow \mathbb{R}$. Dann ist die Menge $\operatorname{lev}_{\leq}^{\alpha}(f, X)$ für alle $\alpha \in \mathbb{R}$ abgeschlossen.	
\end{uebung}

\begin{uebung}
	Für eine abgeschlossene Menge $X \subseteq \mathbb{R}^n$ und endliche Indexmengen $I$ und $J$ seien die Funktion $g_i \colon X \rightarrow \mathbb{R}, i \in I$, und $h_j \colon X \rightarrow \mathbb{R}, j \in J$, stetig. Dann ist die Menge 
		$$ M = \big\{ x \in X ~|~g_i(x) \leq 0, i \in I, ~ h_j(x) = 0, j \in J \big\} $$
		abgeschlossen.	
\end{uebung}

\begin{definition*}
	Die Menge der globalen Minimalpunkte lautet:
	$$ S = \big\{ \overline{x} \in M ~|~ \forall x \in M: f(x) \geq f(\overline{x}) \big\} $$
\end{definition*}

\begin{lemma}
	Für ein $\alpha \in \mathbb{R}$ sei $\operatorname{lev}_{\leq}^{\alpha}(f, M) \neq \emptyset$. Dann gilt 
	$$ S \subseteq \operatorname{lev}_{\leq}^{\alpha}(f, M). $$
\end{lemma}

\begin{satz}[Verschärfter Satz von Weierstraß]
	Für eine (nicht notwendigerweise beschränkte oder abgeschlossene) Menge $M \subseteq \mathbb{R}^n$ sei $f \colon M \rightarrow \mathbb{R}$ stetig, und mit einem $\alpha \in \mathbb{R}$ sei $\operatorname{lev}_{\leq}^{\alpha}(f, M)$ nichtleer und kompakt. Dann besitzt $f$ auf $M$ (mindestens) einen globalen Minimalpunkt.
\end{satz}

\setcounter{thm}{20}

\begin{definition}[Koerzivität]
	Gegeben seien eine abgeschlossene Menge $X \subseteq \mathbb{R}^n$ und eine Funktion $f \colon \mathbb{R}$ fall für alle Folgen $(x^k) \subseteq X$ mit $\lim_k \| x^k \| = +\infty$ auch
	$$ \lim_k f(x^k) = +\infty $$
	gilt, dann heißt $f$ \textbf{koerziv} auf $X$.
\end{definition}

\setcounter{thm}{23}

\begin{uebung}
		Gegeben sei die quadratische Funktion $q(x) = \frac{1}{2} x^T A x + b^T x$ mit einer symmetrischen $(n, n)$-Matrix $A$ (d.h. es gilt $A = A^T$) und $b \in \mathbb{R}^n$. Die Funktion $q$ ist genau dann koerziv auf $\mathbb{R}^n$, wenn $A$ positiv definit ist (d.h. wenn $d^T A d > 0$ für alle $d\in \mathbb{R}^n \setminus \{ 0 \}$ gilt).
\end{uebung}

\begin{beispiel}
	Auf kompakten Mengen $X$ ist jede Funktion $f$ trivialerweise koerziv.	
\end{beispiel}

\begin{lemma}
	Die Funktion $f \colon X \rightarrow \mathbb{R}$ sei stetig und koerziv auf der (nicht notwendigerweise beschränkten) abgeschlossenen Menge $X \subseteq \mathbb{R}^n$. Dann ist die Menge $\operatorname{lev}_{\leq}^{\alpha}(f, X)$ für jedes Niveau $\alpha \in \mathbb{R}$ kompakt.
\end{lemma}

\begin{korollar}
	Es sei $M$ nichtleer und abgeschlossen, aber nicht notwendigerweise beschränkt. Ferner sei die Funktion $f \colon M \rightarrow \mathbb{R}$ stetig und koerziv auf $M$. Dann besitzt $f$ auf $M$ (mindestens) einen globalen Minimalpunkt.
\end{korollar}

\section{Rechenregeln und Umformungen}

\begin{uebung}[Skalare Vielfache und Summen]
	Gegeben seien $M \subseteq \mathbb{R}^n$ und $f,g \colon M \rightarrow \mathbb{R}$. Dann gilt
	\begin{enumerate}[label=\alph*\upshape)]
		\item $\forall \alpha \geq 0$, $\beta \in \mathbb{R} \colon \min_{x \in M} \left( \alpha f(x) + \beta \right) = \alpha \left( \min_{x \in M} f(x) \right) + \beta$
		\item $\forall \alpha <0$, $\beta \in \mathbb{R} \colon \min_{x \in M} \left( \alpha f(x) + \beta \right) = \alpha \left( \max_{x \in M} f(x) \right) + \beta$
		\item $\min_{x \in M} \left( f(x) + g(x) \right) \geq \min_{x \in M} f(x) + \min_{x \in M} g(x)$
	\end{enumerate}
\end{uebung}

\begin{uebung}[Separable Zielfunktion auf kartesischem Produkt]
	Es seien $X \subseteq \mathbb{R}^n$, $Y \subseteq \mathbb{R}^m$, $f \colon X \rightarrow \mathbb{R}$ und $g \colon Y \rightarrow \mathbb{R}$. Dann gilt
	$$ \min_{(x,y) \in X \times Y} \left( f(x) + g(y) \right) = \min_{x \in X} f(x) + \min_{y \in Y} g(y) $$	
\end{uebung}

\begin{uebung}[Vertauschung von Minima und Maxima]
	Es seien $X \subseteq \mathbb{R}^n$, $Y \subseteq \mathbb{R}^m$, $M = X \times Y$ und $f \colon M \rightarrow \mathbb{R}$ gegeben. Dann gilt:
	\begin{enumerate}[label=\alph*\upshape)]
		\item $\min_{(x,y) \in M} f(x, y) = \min_{x \in X} \min_{y \in Y} f(x,y) = \min_{y \in Y} \min_{x \in X} f(x, y)$
		\item $\max_{(x,y) \in M} f(x, y) = \max_{x \in X} \max_{y \in Y} f(x,y) = \max_{y \in Y} \max_{x \in X} f(x, y)$
		\item $\min_{x \in X} \max_{y \in Y} f(x, y) \geq \max_{y \in Y} \min_{x \in X} f(x, y)$
	\end{enumerate}
\end{uebung}

\begin{uebung}[Monotone Transformation]
	Zu $M \subseteq \mathbb{R}^n$ und einer Funktion $f \colon M \rightarrow Y$ mit $Y\subseteq \mathbb{R}$ sei $\psi \colon Y \rightarrow \mathbb{R}$ eine streng monoton wachsende Funktion. Dann gilt
	$$ \min_{x \in M} \psi \left( f(x) \right) = \psi \left( \min_{x \in M} f(x) \right), $$
	und die lokalen bzw. globalen Minimalpunkte stimmen überein.
\end{uebung}

\begin{uebung}[Epigraphumformulierung]
	Gegeben seien $M \subseteq \mathbb{R}^n$ und eine Funktion $f \colon M \rightarrow \mathbb{R}$. Dann sind die Probleme
	$$ P \colon \min_{x \in \mathbb{R}^n} f(x) \text{ s.t. } x \in M ~\text{ und } ~ P_{epi} \colon \min_{(x, \alpha) \in \mathbb{R}^n \times \mathbb{R}} \alpha \text{ s.t. } f(x) \leq \alpha, x \in M $$
	äquivalent, d.h. die Minimalwerte stimmen überein und Minimalpunkte entsprechen sich.
\end{uebung}

\begin{definition}[Parallelprojektion]
	Es sei $M \subseteq \mathbb{R}^n \times \mathbb{R}^m$. Dann heißt
	$$ \operatorname{pr}_x M = \big\{ x \in \mathbb{R}^n ~|~\exists y \in \mathbb{R}^m : (x, y) \in M \big\} $$
	\textbf{Parallelprojektion} von $M$ (den \enquote{$x$-Raum}) $\mathbb{R}^n$.
\end{definition}

\begin{uebung}[Projektionsumformulierung]
	Gegeben seien $M \subseteq \mathbb{R}^n \times \mathbb{R}^m$ und eine Funktion $f \colon \mathbb{R}^n \rightarrow \mathbb{R}$, die nicht von den Variablen aus $\mathbb{R}^m$ abhängt. Dann sind die Probleme 
	$$ P \colon \min_{(x, y) \in \mathbb{R}^n \times \mathbb{R}^m} f(x) ~\text{ s.t.} (x, y) \in M ~\text{ und } ~ P_{proj} \colon \min_{x \in \mathbb{R}^n} f(x) \text{ s.t. } x \in \operatorname{pr}_x M $$
	äquivalent, d.h. die Minimalwerte stimmen überein und Minimalpunkte entsprechen sich.
\end{uebung}

\setcounter{chapter}{2}
\setcounter{section}{0}

\newpage

\section{Optimalitätsbedingungen}

\end{document}