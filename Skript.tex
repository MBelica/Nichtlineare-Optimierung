\documentclass[11pt]{book}

\usepackage{cmap}

\usepackage[utf8]{inputenc}
\usepackage[T1]{fontenc}
\usepackage[english, ngerman]{babel}

\usepackage{microtype}
\usepackage{csquotes}
\usepackage{enumitem}
\usepackage{amsmath}
\usepackage{amsthm}
\usepackage{amsfonts}
\usepackage{mathtools}

\setlength{\parindent}{0cm}
\newcounter{thm}
\numberwithin{thm}{section}

\newtheorem{definition}[thm]{Definition}
\newtheorem{satz}[thm]{Satz}
\newtheorem{uebung}[thm]{Übung}
\newtheorem{lemma}[thm]{Lemma}
\newtheorem*{bemerkung*}{Bemerkung}
\newtheorem*{definition*}{Definition}
\begin{document}

\chapter{Unrestringierte Probleme}

\setcounter{section}{2}

\setcounter{thm}{2}

\begin{definition}[Lösbarkeit] ~\\
	Das Minimierungsproblem $P$ heißt \textbf{lösbar}, falls ein $\overline{x} \in M$ existiert mit
	$$ \inf_{x \in M} f(x) = f(\overline{x}) $$
\end{definition}

\setcounter{thm}{4}

\begin{satz}
	Das Minimierungsproblem $P$ ist genau dann lösbar, wenn es einen globalen Minimalpunkt besitzt.
\end{satz}

\begin{bemerkung*}
	Es können drei Fälle der Unlösbarkeit auftreten:

	\begin{itemize}
		\item $\inf_{x \in M} f(x) = + \infty$
		\item $\inf_{x \in M} f(x) = - \infty$
		\item Ein endliches Infimum wird nicht angenommen.
	\end{itemize}	
\end{bemerkung*}


\begin{satz}[Satz von Weierstraß] ~\\
	Die Menge $M \subseteq \mathbb{R}^n$ sei nichtleer und kompakt, und die Funktion $f \colon M \rightarrow \mathbb{R}$ sei stetig. Dann besitzt $f$ auf $M$ (mindestens) einen globalen Minimalpunkt und einen globalen Maximalpunkt.
\end{satz}

\setcounter{thm}{7}

\begin{definition}[Untere Niveaumenge]
	Für $X \subseteq \mathbb{R}^n, f \colon X \rightarrow \mathbb{R}$ und $\alpha \in \mathbb{R}$ heißt
	$$ \operatorname{lev}_{\leq}^{\alpha}(f, X) = \big\{ x \in X ~|~ f(x) \leq \alpha \big\} $$
	\textbf{untere Niveaumenge von $f$ auf $X$ zum Niveau $\alpha$}. Im Fall $X = \mathbb{R}^n$ schreiben wir auch kurz
	$$ f_{\leq}^{\alpha} \coloneqq \operatorname{lev}_{\leq}^{\alpha}(f, \mathbb{R}^n) = \big\{ x \in \mathbb{R}^n ~|~ f(x) \leq \alpha \big\} $$
\end{definition}

\setcounter{thm}{9}

\begin{uebung}
	Für eine abgeschlossene Menge $X \subseteq \mathbb{R}^n$ sei die Funktion $f \colon X \rightarrow \mathbb{R}$. Dann ist die Menge $\operatorname{lev}_{\leq}^{\alpha}(f, X)$ für alle $\alpha \in \mathbb{R}$ abgeschlossen.	
\end{uebung}

\begin{uebung}
	Für eine abgeschlossene Menge $X \subseteq \mathbb{R}^n$ und endliche Indexmengen $I$ und $J$ seien die Funktion $g_i \colon X \rightarrow \mathbb{R}, i \in I$, und $h_j \colon X \rightarrow \mathbb{R}, j \in J$, stetig. Dann ist die Menge 
		$$ M = \big\{ x \in X ~|~g_i(x) \leq 0, i \in I, ~ h_j(x) = 0, j \in J \big\} $$
		abgeschlossen.	
\end{uebung}

\begin{definition*}
	Die Menge der globalen Minimalpunkte lautet:
	$$ S = \big\{ \overline{x} \in M ~|~ \forall x \in M: f(x) \geq f(\overline{x}) \big\} $$
\end{definition*}

\begin{lemma}
	Für ein $\alpha \in \mathbb{R}$ sei $\operatorname{lev}_{\leq}^{\alpha}(f, M) \neq \emptyset$. Dann gilt 
	$$ S \subseteq \operatorname{lev}_{\leq}^{\alpha}(f, M). $$
\end{lemma}

\begin{satz}[Verschärfter Satz von Weierstraß]
	Für eine (nicht notwendigerweise beschränkte oder abgeschlossene) Menge $M \subseteq \mathbb{R}^n$ sei $f \colon M \rightarrow \mathbb{R}$ stetig, und mit einem $\alpha \in \mathbb{R}$ sei $\operatorname{lev}_{\leq}^{\alpha}(f, M)$ nichtleer und kompakt. Dann besitzt $f$ auf $M$ (mindestens) einen globalen Minimalpunkt.
\end{satz}

\end{document}